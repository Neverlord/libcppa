\section{Pattern Matching}
\label{Sec::PatternMatching}

Actor programming implies a message passing paradigm.
This means that defining message handlers is a recurring task.
The easiest and most natural way to specify such message handlers is pattern matching.
Unfortunately, C++ does not provide any pattern matching facilities.
%A general pattern matching solution for arbitrary data structures would require a language extension.
Hence, we provide an internal domain-specific language to match incoming messages.

\subsection{Basics}
\label{Sec::PatternMatching::Basics}

A match expression begins with a call to the function \lstinline^on^, which returns an intermediate object providing \lstinline^operator>>^.
The right-hand side of the operator denotes a callback, usually a lambda expression, that should be invoked if a tuple matches the types given to \lstinline^on^,
as shown in the example below.

\begin{lstlisting}
on<int>() >> [](int i) { /*...*/ }
on<int, float>() >> [](int i, float f) { /*...*/ }
on<int, int, int>() >> [](int a, int b, int c) { /*...*/ }
\end{lstlisting}

The result of \lstinline^operator>>^ is a \textit{match statement}.
A message handler can consist of any number of match statements.
At most one callback is invoked, since the evaluation stops at the first match.

\begin{lstlisting}
message_handler fun {
  on<int>() >> [](int i) {
    // case1
  },
  on<int>() >> [](int i) {
    // case2; never invoked, since case1 always matches first
  }
};
\end{lstlisting}

The function ``\lstinline^on^'' can be used in two ways.
Either with template parameters only or with function parameters only.
The latter version deduces all types from its arguments and matches for both type and value.
To match for any value of a given type, ``\lstinline^val^'' can be used, as shown in the following example.

\begin{lstlisting}
on(42) >> [](int i) { assert(i == 42); }
on("hello world") >> [] { /* ... */ }
on("print", val<std::string>) >> [](const std::string& what) {
  // ...
}
\end{lstlisting}

\textbf{Note:} The given callback can have less arguments than the pattern.
But it is only allowed to skip arguments from left to right.

\begin{lstlisting}
on<int, float, double>() >> [](double) { /*...*/ }             // ok
on<int, float, double>() >> [](float, double) { /*...*/ }      // ok
on<int, float, double>() >> [](int, float, double) { /*...*/ } // ok

on<int, float, double>() >> [](int i) { /*...*/ } // compiler error
\end{lstlisting}

\subsection{Reducing Redundancy with ``\lstinline^arg_match^'' and ``\lstinline^on_arg_match^''}

Our previous examples always used the most verbose form, which is quite redundant, since you have to type the types twice -- as template parameter and as argument type for the lambda.
To avoid such redundancy, \lstinline^arg_match^ can be used as last argument to the function \lstinline^on^.
This causes the compiler to deduce all further types from the signature of the given callback.

\begin{lstlisting}
on<int, int>() >> [](int a, int b) { /*...*/ }
// is equal to:
on(arg_match) >> [](int a, int b) { /*...*/ }
\end{lstlisting}

Note that the second version does call \lstinline^on^ without template parameters.
Furthermore, \lstinline^arg_match^ must be passed as last parameter.
If all types should be deduced from the callback signature, \lstinline^on_arg_match^ can be used.
It is equal to \lstinline^on(arg_match)^.
However, when using a pattern to initialize the behavior of an actor, \lstinline^on_arg_match^ is used implicitly whenever a functor is passed without preceding it with an \lstinline^on^ clause.

\begin{lstlisting}
on_arg_match >> [](const std::string& str) { /*...*/ }
\end{lstlisting}

\subsection{Atoms}
\label{Sec::PatternMatching::Atoms}

Assume an actor provides a mathematical service for integers.
It takes two arguments, performs a predefined operation and returns the result.
It cannot determine an operation, such as multiply or add, by receiving two operands.
Thus, the operation must be encoded into the message.
The Erlang programming language introduced an approach to use non-numerical
constants, so-called \textit{atoms}, which have an unambiguous, special-purpose type and do not have the runtime overhead of string constants.
Atoms are mapped to integer values at compile time in \lib.
This mapping is guaranteed to be collision-free and invertible, but limits atom literals to ten characters and prohibits special characters.
Legal characters are ``\lstinline[language=C++]^_0-9A-Za-z^'' and the whitespace character.
Atoms are created using the \lstinline^constexpr^ function \lstinline^atom^, as the following example illustrates.

\begin{lstlisting}
on(atom("add"), arg_match) >> [](int a, int b) { /*...*/ },
on(atom("multiply"), arg_match) >> [](int a, int b) { /*...*/ },
// ...
\end{lstlisting}

\textbf{Note}: The compiler cannot enforce the restrictions at compile time, except for a length check.
The assertion \lstinline^atom("!?") != atom("?!")^ is not true, because each invalid character is mapped to the whitespace character.

\clearpage
\subsection{Wildcards}
\label{Sec::PatternMatching::Wildcards}

The type \lstinline^anything^ can be used as wildcard to match any number of any types.
%The \lstinline^constexpr^ value \lstinline^any_vals^ can be used as function argument if \lstinline^on^ is used without template paremeters.
A pattern created by \lstinline^on<anything>()^ or its alias \lstinline^others()^ is useful to define a default case.
For patterns defined without template parameters, the \lstinline^constexpr^ value \lstinline^any_vals^ can be used as function argument.
The constant \lstinline^any_vals^ is of type \lstinline^anything^ and is nothing but syntactic sugar for defining patterns.

\begin{lstlisting}[language=C++]
on<int, anything>() >> [](int i) {
  // tuple with int as first element
},
on(any_vals, arg_match) >> [](int i) {
  // tuple with int as last element
  // "on(any_vals, arg_match)" is equal to "on(anything{}, arg_match)"
},
others() >> [] {
  // everything else (default handler)
  // "others()" is equal to "on<anything>()" and "on(any_vals)"
}
\end{lstlisting}

\subsection{Projections and Extractors}

Projections perform type conversions or extract data from a given input.
If a callback expects an integer but the received message contains a string, a projection can be used to perform a type conversion on-the-fly.
This conversion should be free of side-effects and, in particular, shall not throw exceptions, because a failed projection is not an error.
A pattern simply does not match if a projection failed.
Let us have a look at a simple example.

\begin{lstlisting}
auto intproj = [](const string& str) -> option<int> {
  char* endptr = nullptr;
  int result = static_cast<int>(strtol(str.c_str(), &endptr, 10));
  if (endptr != nullptr && *endptr == '\0') return result;
  return {};
};
message_handler fun {
  on(intproj) >> [](int i) {
    // case 1: successfully converted a string
  },
  [](const string& str) {
    // case 2: str is not an integer
  }
};
\end{lstlisting}

The lambda \lstinline^intproj^ is a \lstinline^string^ $\Rightarrow$ \lstinline^int^ projection, but note that it does not return an integer.
It returns \lstinline^option<int>^, because the projection is not guaranteed to always succeed.
An empty \lstinline^option^ indicates, that a value does not have a valid mapping to an integer.
A pattern does not match if a projection failed.

\textbf{Note}: Functors used as projection must take exactly one argument and must return a value.
The types for the pattern are deduced from the functor's signature.
If the functor returns an \lstinline^option<T>^, then \lstinline^T^ is deduced.

\subsection{Dynamically Building Messages}

Usually, messages are created implicitly when sending messages but can also be created explicitly using \lstinline^make_message^.
In both cases, types and number of elements is fixed at compile time.
To allow for fully dynamic message generation, \lib also offers a third option to create messages by using a \lstinline^message_builder^:

\begin{lstlisting}
message_builder mb;
// prefix message with some atom
mb.append(atom("strings"));
// fill message with some strings
std::vector<std::string> strings{/*...*/};
for (auto& str : strings) {
  mb.append(str);
}
// create the message
message msg = mb.to_message();
\end{lstlisting}
