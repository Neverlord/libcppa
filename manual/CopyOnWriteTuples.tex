\section{Copy-On-Write Tuples}
\label{Sec::Tuples}

The message passing implementation of \libcppa uses tuples with call-by-value semantic.
Hence, it is not necessary to declare message types, though, \libcppa allows users to use user-defined types in messages (see Section \ref{Sec::TypeSystem::UserDefined}).
A call-by-value semantic would cause multiple copies of a tuple if it is send to multiple actors.
To avoid unnecessary copying overhead, \libcppa uses a copy-on-write tuple implementation.
A tuple is implicitly shared between any number of actors, as long as all actors demand only read access.
Whenever an actor demands write access, it has to copy the data first if more than one reference to it exists.
Thus, race conditions cannot occur and each tuple is copied only if necessary.

The interface of \lstinline^cow_tuple^ strictly distinguishes between const and non-const access.
The template function \lstinline^get^ returns an element as immutable value, while \lstinline^get_ref^ explicitly returns a mutable reference to the required value and detaches the tuple if needed.
We do not provide a const overload for \lstinline^get^, because this would cause to unintended, and thus unnecessary, copying overhead.

\begin{lstlisting}
auto x1 = make_cow_tuple(1, 2, 3);     // cow_tuple<int, int, int>
auto x2 = x1;                          // cow_tuple<int, int, int>
assert(&get<0>(x1) == &get<0>(x2));    // point to the same data
get_ref<0>(x1) = 10;                   // detaches x1 from x2
//get<0>(x1) = 10;                     // compiler error
assert(get<0>(x1) == 10);              // x1 is now {10, 2, 3}
assert(get<0>(x2) == 1);               // x2 is still {1, 2, 3}
assert(&get<0>(x1) != &get<0>(x2));    // no longer the same
\end{lstlisting}

\subsection{Dynamically Typed Tuples}
\label{Sec::Tuples::DynamicallyTypedTuples}

The class \lstinline^any_tuple^ represents a tuple without static type information.
All messages send between actors use this tuple type.
The type information can be either explicitly accessed for each element or the original tuple, or a subtuple of it, can be restored using \lstinline^tuple_cast^.
Users of \libcppa usually do not need to know about
\lstinline^any_tuple^, since it is used ``behind the scenes''.
However, \lstinline^any_tuple^ can be created from a \lstinline^cow_tuple^
or by using \lstinline^make_any_tuple^, as shown below.

\begin{lstlisting}
auto x1 = make_cow_tuple(1, 2, 3);     // cow_tuple<int, int, int>
any_tuple x2 = x1;                     // any_tuple
any_tuple x3 = make_cow_tuple(10, 20); // any_tuple
auto x4 = make_any_tuple(42);          // any_tuple
\end{lstlisting}

\clearpage
\subsection{Casting Tuples}

The function \lstinline^tuple_cast^ restores static type information
from an \lstinline^any_tuple^ object.
It returns an \lstinline^option^ (see Section \ref{Appendix::Option})
for a \lstinline^cow_tuple^ of the requested types.

\begin{lstlisting}
auto x1 = make_any_tuple(1, 2, 3);
auto x2_opt = tuple_cast<int, int, int>(x1);
assert(x2_opt.valid());
auto x2 = *x2_opt;
assert(get<0>(x2) == 1);
assert(get<1>(x2) == 2);
assert(get<2>(x2) == 3);
\end{lstlisting}

The function \lstinline^tuple_cast^ can be used with wildcards (see Section \ref{Sec::PatternMatching::Wildcards}) to create a view to a subset of the original data.
No elements are copied, unless the tuple becomes detached.

\begin{lstlisting}
auto x1 = make_cow_tuple(1, 2, 3);
any_tuple x2 = x1;
auto x3_opt = tuple_cast<int, anything, int>(x2);
assert(x3_opt.valid());
auto x3 = *x3_opt;
assert(get<0>(x3) == 1);
assert(get<1>(x3) == 3);
assert(&get<0>(x3) == &get<0>(x1));
assert(&get<1>(x3) == &get<2>(x1));
\end{lstlisting}
