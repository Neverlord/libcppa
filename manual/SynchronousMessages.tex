\section{Synchronous Communication}
\label{Sec::Sync}

\libcppa uses a future-based API for synchronous communication.
The functions \lstinline^sync_send^ and \lstinline^sync_send_tuple^ send synchronous request messages to the receiver and return a future to the response message.
Note that the returned future is \textit{actor-local}, i.e., only the actor that has send the corresponding request message is able to receive the response identified by such a future.

\begin{lstlisting}
template<typename... Args>
message_future sync_send(actor_ptr whom, Args&&... what);

message_future sync_send_tuple(actor_ptr whom, any_tuple what);

template<typename Duration, typename... Args>
message_future timed_sync_send(actor_ptr whom,
                               Duration timeout,
                               Args&&... what);

template<typename Duration, typename... Args>
message_future timed_sync_send_tuple(actor_ptr whom,
                                     Duration timeout,
                                     any_tuple what);
\end{lstlisting}

A synchronous message is sent to the receiving actor's mailbox like any other asynchronous message.
The response message, on the other hand, is treated separately.

The difference between \lstinline^sync_send^ and \lstinline^timed_sync_send^ is how timeouts are handled.
The behavior of \lstinline^sync_send^ is analogous to \lstinline^send^, i.e., timeouts are specified by using \lstinline^after(...)^ statements (see \ref{Sec::Receive::Timeouts}).
When using \lstinline^timed_sync_send^ function, \lstinline^after(...)^ statements are ignored and the actor will receive a \lstinline^'TIMEOUT'^ message after the given duration instead.

\subsection{Error Messages}

When using synchronous messaging, \libcppa's runtime environment will send ...

\begin{itemize}
\item \lstinline^{'EXITED', uint32_t exit_reason}^ if the receiver is not alive
\item \lstinline^{'VOID'}^ if the receiver handled the message but did not respond to it
\item \lstinline^{'TIMEOUT'}^ if a message send by \lstinline^timed_sync_send^ timed out
\end{itemize}

\clearpage
\subsection{Receive Response Messages}

The function \lstinline^handle_response^ can be used to set a one-shot handler receiving the response message send by \lstinline^sync_send^.

\begin{lstlisting}
actor_ptr testee = ...; // replies with a string to 'get'

// "handle_response" usage example
auto handle = sync_send(testee, atom("get"));
handle_response (handle) (
  on_arg_match >> [=](const std::string& str) {
    // handle str
  },
  after(std::chrono::seconds(30)) >> [=]() {
    // handle error
  }
);
\end{lstlisting}

Similar to \lstinline^become^, the function \lstinline^handle_response^ modifies an actor's behavior stack.
However, it is used as ``one-shot handler'' and automatically returns the previous actor behavior afterwards.
It is possible to ``stack'' multiple \lstinline^handle_response^ calls.
Each response handler is executed once and then automatically discarded.

\subsection{Synchronous Failures and Error Handlers}

An unexpected response message, i.e., a message that is not handled by given behavior, will invoke the actor's \lstinline^on_sync_failure^ handler.
The default handler kills the actor by calling \lstinline^self->quit(exit_reason::unhandled_sync_failure)^.
The handler can be overridden by calling \lstinline^self->on_sync_failure(/*...*/)^.

Unhandled \lstinline^'TIMEOUT'^ messages trigger the \lstinline^on_sync_timeout^ handler.
The default handler kills the actor for reason \lstinline^exit_reason::unhandled_sync_failure^.
It is possible set both error handlers by calling \lstinline^self->on_sync_timeout_or_failure(/*...*)^.


\clearpage
\subsection{Using \lstinline^then^ to Receive a Response}

Often, an actor sends a synchronous message and then wants to wait for the response.
In this case, using either \lstinline^handle_response^ is quite verbose.
To allow for a more compact code, \lstinline^message_future^ provides the member function \lstinline^then^.
Using this member function is equal to using \lstinline^handle_response^, as illustrated by the following example.

\begin{lstlisting}
actor_ptr testee = ...; // replies with a string to 'get'

// set handler for unexpected messages
self->on_sync_failure = [] {
    aout << "received: " << to_string(self->last_dequeued()) << endl;
};

// set handler for timeouts
self->on_sync_timeout = [] {
    aout << "timeout occured" << endl;
};

// set response handler by using "then"
timed_sync_send(testee, std::chrono::seconds(30), atom("get")).then(
  on_arg_match >> [=](const std::string& str) { /* handle str */ }
);
\end{lstlisting}

\subsubsection{Using Functors without Patterns}

To reduce verbosity, \libcppa supports synchronous response handlers without patterns.
In this case, the pattern is automatically deduced by the functor's signature.

\begin{lstlisting}
actor_ptr testee = ...; // replies with a string to 'get'

// (1) functor only usage
sync_send(testee, atom("get")).then(
  [=](const std::string& str) { /*...*/ }
);
// statement (1) is equal to:
sync_send(testee, atom("get")).then(
  on(any_vals, arg_match) >> [=](const std::string& str) { /*...*/ }
);
\end{lstlisting}

\clearpage
\subsubsection{Continuations for Event-based Actors}

\libcppa supports continuations to enable chaining of send/receive statements.
The functions \lstinline^handle_response^ and \lstinline^message_future::then^ both return a helper object offering the member function \lstinline^continue_with^, which takes a functor $f$ without arguments.
After receiving a message, $f$ is invoked if and only if the received messages was handled successfully, i.e., neither \lstinline^sync_failure^ nor \lstinline^sync_timeout^ occurred.

\begin{lstlisting}
actor_ptr d_or_s = ...; // replies with either a double or a string

sync_send(d_or_s, atom("get")).then(
  [=](double value) { /* functor f1 */ },
  [=](const string& value) { /* functor f2*/ }
).continue_with([=] {
  // this continuation is invoked in both cases
  // *after* f1 or f2 is done, but *not* in case
  // of sync_failure or sync_timeout
});
\end{lstlisting}
